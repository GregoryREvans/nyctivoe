\documentclass[11pt]{article}
\usepackage{fontspec}
\usepackage[utf8]{inputenc}
\usepackage{xunicode}
\setmainfont{Bell MT}
\usepackage[paperwidth=8.5in,paperheight=11in,margin=1in,headheight=0.0in,footskip=0.5in,includehead,includefoot,portrait]{geometry}
\usepackage[absolute]{textpos}
\TPGrid[0.5in, 0.25in]{23}{24}
\parindent=0pt
\parskip=12pt
\usepackage{nopageno}
\usepackage{graphicx}
\graphicspath{ {./images/} }
\usepackage{amsmath}
\usepackage{tikz}
\newcommand*\circled[1]{\tikz[baseline=(char.base)]{
            \node[shape=circle,draw,inner sep=1pt] (char) {#1};}}

\begin{document}

\begin{textblock}{23}(0, 1)
\begin{center}
\huge FOREWORD
\end{center}
\end{textblock}

\vspace*{0.25\baselineskip}

\begingroup

\begin{center}
\leftskip0.25in
Nyctivoe is a compound word comprised of the Greek words for cry or shout and night. A nyctivoe could refer to ceremonial calls to the moon goddess.
\rightskip\leftskip
\phantom{text} \hfill \phantom{()}
\end{center}

\endgroup

\vspace*{5\baselineskip}

\begin{center}
\huge INSTRUMENTATION
\end{center}

\hspace*{1cm} Tenor Saxophone
\\
\hspace*{1cm} Baritone Saxophone
\\
\hspace*{1cm} Percussion
\\
\hspace*{2cm} Instruments:
\\
\hspace*{3cm} Brake Drum + stone (always scrape brake drum with stone)
\\
\hspace*{3cm} Bass Drum
\\
\hspace*{3cm} Log Drums [x4 pitches]
\\
\hspace*{3cm} ``Gongs'' [x4] (preferably 1 large tam tam, 1 medium tam tam, 1 large gong, 1 medium gong)
\\
\hspace*{3cm} Low Timpani + Cymbal, inverted, resting on skin (always strike cymbal)
\\
\hspace*{2cm} Implements:
\\
\hspace*{3cm} Mallets suitable for all drums
\\
\hspace*{3cm} Superball Mallet
\\
\hspace*{1cm} Viola
\\
\hspace*{1cm} Violoncello

\vspace*{1.25\baselineskip}

\begin{center}
\huge PERFORMANCE NOTES
\end{center}


\begin{center}
\huge Saxophones
\end{center}
\begingroup
\begin{center}

\leftskip0.25in
\pmb{Son fendu} : Split-tone / overtone multiphonics for the Baritone Saxophone are notated with a green underlay. The width of the underlay represents the height of partials achieved.
\rightskip\leftskip
\phantom{text} \hfill \phantom{()}

\leftskip0.25in
\pmb{Slap Tongue} : is notated with an accent note head.
\rightskip\leftskip
\phantom{text} \hfill \phantom{()}

\leftskip0.25in
\pmb{Key Clicks} : are notated with X noteheads.
\rightskip\leftskip
\phantom{text} \hfill \phantom{()}

\leftskip0.25in
\pmb{Singing + Playing} : Sometimes vocalization while playing is notated on an auxiliary staff of three lines. The vocalizations should be in the modal register of the voice.
\rightskip\leftskip
\phantom{text} \hfill \phantom{()}

\leftskip0.25in
\pmb{Miscellaneous} : \circled{1} Diamond note heads represent a very airy tone. \circled{2} Half-airy tone is shown with a diamond half-filled with black for short durations and a diamond open on one end for long durations.
\rightskip\leftskip
\phantom{text} \hfill \phantom{()}

\end{center}
\endgroup

\begin{center}
\huge Strings
\end{center}
\begingroup
\begin{center}

\leftskip0.25in
\pmb{String Contact Points} : The indications of string contact positions such as $sul \ tasto$ (abbreviated as $T$), $sul \ ponticello$ (abbreviated as $P$), $extreme \ sul \ tasto$ (abbreviated as $XT$), etc. should be considered as points along the continuum of the length string. The performer should make an effort to smoothly transition from one position to the next throughout the duration of the passage covered by the arrow-demarcated dashed line. When this arrow is not present, the performer should default to an $ordinario$ position. Sometimes an auxiliary staff appears above to indicated position changes. Dashed lines represent the tasto region (range ad lib) and solid lines represent the ordinario to ponticello region.
\rightskip\leftskip
\phantom{text} \hfill \phantom{()}

\leftskip0.25in
\pmb{Bow Contact Points} : In various passages throughout this piece, there is notation which represents the point at which the bow is touched as it is drawn across the string. These positions are written as fractions where \( \frac{0}{7} \) and  \( \frac{0}{5} \) represent $au \ talon$ and \( \frac{7}{7} \) and \( \frac{5}{5} \) represent $punta \ d'arco$. For the duration of the note to which these fractions are attached, the performer should draw the bow at a constant speed, moving toward the destination point indicated on the following note. Bowings are provided. Passages without these indications should be bowed at the performer's discretion.
\rightskip\leftskip
\phantom{text} \hfill \phantom{()}

\leftskip0.25in
\pmb{Bow Rotation Indications} : \circled{1} $col \ legno \ tratto$ is abbreviated as $clt.$ and \circled{2} $col \ legno \ batutto$ is abbreviated as $clb.$. When these abbreviations are not present, the performer should default to ordinary $crine$ bowing techniques.
\rightskip\leftskip
\phantom{text} \hfill \phantom{()}

\leftskip0.25in
\pmb{Spazzolato} : is notated with an arrow attached to the stems with the bowing direction indicated by the angle of the arrow.
\rightskip\leftskip
\phantom{text} \hfill \phantom{()}

\leftskip0.25in
\pmb{String Contact Points} : The indications of string contact positions such as $sul \ tasto$ (abbreviated as $T$), $sul \ ponticello$ (abbreviated as $P$), $extreme \ sul \ tasto$ (abbreviated as $XT$), etc. should be considered as points along the continuum of the length string. The performer should make an effort to smoothly transition from one position to the next throughout the duration of the passage covered by the arrow-demarcated dashed line. When this arrow is not present, the performer should default to an $ordinario$ position.
\rightskip\leftskip
\phantom{text} \hfill \phantom{()}

\leftskip0.25in
\pmb{String Crossing} : is sometimes notated on a four line auxiliary staff.
\rightskip\leftskip
\phantom{text} \hfill \phantom{()}

\leftskip0.25in
\pmb{Miscellaneous} : \circled{1} Tremoli should be performed as fast as possible and not as a measured subdivision of the duration to which they are attached. \circled{2} Diamond note heads represent a left hand finger pressure of a natural harmonic. \circled{3} Half-harmonic finger pressure is shown with a diamondhalf-filled with black for short durations and a diamond open on one end for long durations.
\rightskip\leftskip
\phantom{text} \hfill \phantom{()}

\end{center}
\endgroup

\begin{center}
\huge All
\end{center}
\begingroup
\begin{center}

\leftskip0.25in
\pmb{Accidentals} : After temporary accidentals, cancellation marks are printed also in the following measure (for notes in the same octave) and, in the same measure, for notes in other octaves, but they are printed again if the same note appears later in the same measure, except if the note is immediately repeated.
\rightskip\leftskip
\phantom{text} \hfill \phantom{()}

\end{center}
\endgroup

\vspace*{9\baselineskip}

\begin{center}
\textit{Nyctivoe} was composed for the Steph Tamas.
\end{center}

\vspace*{23\baselineskip}

\begin{center}
duration: c. 13'
\end{center}

\end{document}
